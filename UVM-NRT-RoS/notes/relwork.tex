\section{Background: ML Algorithms and Feature Extraction}

A variety of recent work has studied classification of environmental
sounds using machine learning analysis of audio data collected by
embedded devices \cite{electronics11223743,Mu2021,s22093504},
including detection of rain
\cite{quteprints82848,Ferroudj_et_al.,Guico_et_al.,Avanzato_et_al.}. This
previous work has considered a menagerie of ML algorithms, from SVMs and
decision tree architectures
\cite{s22093504,quteprints82848,Guico_et_al.}  to neural networks
\cite{Avanzato_et_al.,Mu2021}, as well as statistical (e.g., Bayesian)
analysis \cite{s22093504,Avanzato_et_al.}.

\subsection{Feature Extraction}
\label{relwork-features}

Prior to algorithm choices, a primary consideration in any machine
learning analysis of audio data is feature extraction. Electronic
formats for sound- e.g., .wav- record audio as time series data.
These recordings themselves and other aggregate data directly
extracted from it, such as zero crossing rate (ZCR) and root mean
squared energy (RMSE), are generally termed \emph{time domain} audio
data.  Time domain data is also commonly cast into the \emph{frequency
domain} as a \emph{spectogram} via Fourier transformation, especially as a Mel
spectogram. Additional manipulation of the Mel spectogram obtains
frequency domain features in the form of Mel-frequency cepstral
coefficients (MFCCs), which are a dominant feature in ML analysis of
audio data.

Both time and frequency domain audio features have been used in
previous related work. Features of most effective models for rainfall
detection
\cite{Avanzato_et_al.,Avanzato_Fransesco_et_al.,quteprints82848}
include the following:
\begin{enumerate}[\hspace{5mm}F1.]
\item Mel-frequency Cepstral Coefficients (MFCCs)
\item First Autocorrelation Coefficient (ACC)
\item Zero Crossing Rate (ZCR)
\item Temporal and Spectoral Entropy Indices (H$_t$ and H$_f$)
\item Acoustic Complexity Index (ACI)
\item Background Noise (BgN)
\item Spectral Cover (SC)
\end{enumerate}
The way these features are used varies significantly depending on
the approach. 
A variety of work has focused on MFCCs, especially neural network
based approaches \cite{Avanzato_et_al.,Avanzato_Fransesco_et_al.}.
Approaches using simpler modeling algorithms, such as SVMs and
Decision trees, have used F3-F7 \cite{quteprints82848}. And statistical
analysis in previous work ``shows the possibility of being able to
clearly discriminate all levels of rainfall intensity through the
parameters of zero crossing rate and first autocorrelation
coefficient'' (F1 and F2) \cite{Avanzato_et_al.}. Some related work in
environmental sound classification has also considered sophisticated
combinations of time- and frequency-domain features , so-called joint
signal analysis, to obtain better model performance
\cite{electronics11223743}.

Some more recent work has leveraged the power of deep learning to more
directly and automatically extract features from ``raw''
spectograms. This includes approaches to rain detection where the
spectogram itself is the feature set \cite{Avanzato_et_al.}, and
approaches to general environemental sound classification where neural
networks are used to automatically extract features from spectograms
\cite{electronics11223743}.

\subsection{ML Algorithms and Performance}

Earlier work on rain detection from acoustic signals considered simple
ML algorithms such as SVMs and decision trees \cite{quteprints82848}.
Later work, both in rain detection and environmental sound detection
more generally, has considered ensemble methods and boosting
\cite{electronics11223743,s22093504}. As in many other areas,
interest has recently grown in neural network approaches to modeling
\cite{Avanzato_et_al.,Avanzato_Fransesco_et_al.,electronics11223743},
particularly CNNs.

Interestingly, none of these methods seems to significantly outperform
the earliest work on rainfall prediction \cite{quteprints82848} using
SVM classifiers and features F3-F7, where accuracy of 93\% was
achieved, though competitive baseline performance metrics have been
reported for essentially all of the previous work discussed so
far. More recent work using neural network models has made some
contextual advancements, for example successful integration with a
low-cost, low-power embedded system \cite{Avanzato_Fransesco_et_al.},
and rain detection in challenging urban environments \cite{s22093504}.

