\section{Research Plan: ML Algorithms and Feature Extraction}

Previous approaches to predicting rain using ML analysis of audio data
show that the general approach is promising. However, our intended
application has several novel aspects and system constraints. Previous
related approaches have considered detection of rain events, or rain
events amidst a host of other sound classifications. And while there
has been previous work on the integration rain detection analysis with
low-cost embedded systems for audio data collection
\cite{Avanzato_Fransesco_et_al.,Guico_et_al.}, these works have not
considered ML in the edge- i.e., embedding the full process of data
collection, feature extraction, and model prediction on a low-cost,
low-powered platform.

Additionally, we are interested spcifically in the more general
problem of precipation phase detection. And we are interested in
operating in remote, alpine settings.

Altogether, the characteristics of our application space recommend
the use of simpler ML models. The Arduino platform can easily support
implementations of decision trees, including ensembles and even
boosting. However implementation of neural networks is currently a
challenge. While TensorFlow Lite has been developed for edge devices,
for example, it is only available on one Arduino platform that is
not available for purchase at the time of this writing. 



