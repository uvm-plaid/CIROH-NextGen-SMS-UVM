\section{Research Plan: ML Algorithms and Feature Extraction}

Previous approaches to predicting rain using ML analysis of audio data
show that the general approach is promising. However, our intended
application has several novel aspects and system constraints. Previous
related approaches have considered detection of rain events, or rain
events amidst a host of other sound classifications in urban settings.
But we are interested in the more general problem of precipation phase
detection, and operating in remote, alpine settings. And while there
has been previous work on the integration rain detection analysis with
low-cost embedded systems for audio data collection
\cite{Avanzato_Fransesco_et_al.,Guico_et_al.}, these works have not
considered ML in the edge- i.e., embedding the full process of data
collection, feature extraction, and model classification on a low-cost,
low-powered platform. We hypothesize that implementing the
full detection workflow on-device will
be more feasible than communicating streams of audio data
over radio for analysis in the system back end.

Thus, with respect to ML algorithm and feature selection, we can't adapt
existing models for precipation phase detection
from acoustic data. And the computational and power characteristics of our
application space recommend the use of simpler ML models. The Arduino
platform can easily support implementations of decision trees,
including ensembles and even boosting. However implementation of
neural networks in low-cost embedded systems is currently a
challenge. TensorFlow Lite has been developed for edge devices, for
example, but it is only available on one Arduino platform that is not
available for purchase at the time of this writing. While
implementation on, e.g., embedded Linux platforms would be feasible,
this would entail an order of magnitude greater power consumption.

\subsection{Machine Learning Experiments}

We will evaluate SVM, Random Forest (RF), and boosted decision tree
(XGBoost) architectures, as well as ensembles of these architectures.
We will focus on models using features F1-F7. We evaluate both feature
importance of F1-F7, and evaluate performance of models in terms of
accuracy and ROC metrics. In our initial exploration we will use
the dataset we developed for simulated rainfall and sleet at different
intensities.
