\section{Research Plan: ML Algorithms and Feature Extraction}

Previous approaches to predicting rain using ML analysis of audio data
show that the general approach is promising. However, our intended
application has several novel aspects and system constraints. Previous
related approaches have considered detection of rain events, or rain
events amidst a host of other sound classifications. And while there
has been previous work on the integration rain detection analysis with
low-cost embedded systems for audio data collection
\cite{Avanzato_Fransesco_et_al.,Guico_et_al.}, these works have not
considered ML in the edge- i.e., embedding the full process of data
collection, feature extraction, and model prediction on a low-cost,
low-powered platform.

Additionally, we are interested spcifically in the more general
problem of precipation phase detection. And we are interested in
operating in remote, alpine settings. So, our ML problem has not
been directly studied previously, though earlier work provides
confidence in the general approach. 

Thus, it is necessary for us to consider our problem ``from scratch''
and not simply adapt existing models. And the computational and power
characteristics of our application space recommend the use of simpler
ML models. The Arduino platform can easily support implementations of
decision trees, including ensembles and even boosting. However
implementation of neural networks in low-cost embedded systems is
currently a challenge. TensorFlow Lite has been developed for edge
devices, for example, but it is only available on one Arduino platform
that is not available for purchase at the time of this writing. While
implementation on, e.g., embedded Linux platforms would be feasible,
this would entail an order of magnitude greater power consumption.

\subsection{Machine Learning Experiments}

We will evaluate SVM, Random Forest (RF), and boosted decision tree
(XGBoost) architectures, as well as ensembles of these architectures.
We will focus on models using features F1-F7. We evaluate both feature
importance of F1-F7, and evaluate performance of models in terms of
accuracy and ROC metrics. In our initial exploration we will use
the dataset we developed for simulated rainfall and sleet at different
intensities.
